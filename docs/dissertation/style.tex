\usepackage[left=1.5in, top=1.4in, bottom=1.4in, right=1.1in]{geometry}
\usepackage{phdthesis}
\usepackage{caption}
\usepackage{epsfig,titlesec,amsfonts,amsthm,amssymb,amsmath,setspace,enumerate,pxfonts,txfonts}
\usepackage[font=footnotesize]{subfig}
\usepackage{setspace}
\usepackage[T1]{fontenc}
\usepackage[utf8]{inputenc}
\usepackage[english]{babel}
\usepackage{mathrsfs}
\usepackage{tabularx}
\usepackage{color, colortbl}
\definecolor{Gray}{gray}{0.9}
\usepackage{wrapfig}
\usepackage{cite}
\usepackage{url}
\usepackage{multirow}
\usepackage[nochapter]{vhistory}
\usepackage[hidelinks]{hyperref}
\usepackage{longtable}
\usepackage[nonumberlist]{glossaries}
\usepackage{soul}
\usepackage{enumitem}
\usepackage{algorithm}
\usepackage{algorithmic}
\usepackage{todonotes}
\usepackage{amsthm}
\usepackage{bibentry}
\usepackage[subfigure,titles]{tocloft} % for editing TOC (add Chapter and Appendix)
\nobibliography*
%define todoin as cmd to print todo note inline and not at margins
\newcommand\todoin[2][]{\todo[inline, caption={2do}, #1]{
\begin{minipage}{\textwidth-4pt}#2\end{minipage}}}

%\usepackage[finalnew]{trackchanges}
% Track changes has a number of different ways that it can display the
% edits in the final dvi or pdf file.
%   finalold  - Reject all edits.
%   finalnew  - Accept all edits.
%   footnotes - Display edits as footnotes.
%   margins   - Display edits as margin notes.
%   inline    - Display edits inline. 
%   \note[editor]{The note}
%   \annote[editor]{Text to annotate}{The note}
%   \add[editor]{Text to add}
%   \remove[editor]{Text to remove}
%   \change[editor]{Text to remove}{Text to add} 

% Trackchanges
%\addeditor{CT}
%\addeditor{JB}

\makeglossaries

\makeatletter
\newtheoremstyle{researchTopic}% name of the style to be used
  {25pt}% space before
  {5pt}% space after
  {}% body font
  {}% indent
  {\bfseries}% header font
  {.}% punctuation
  {.5em}% after theorem header
  {}% header specification (empty for default)
\makeatother

\makeatletter
\newtheoremstyle{indented}% name of the style to be used
  {15pt}% space before
  {5pt}% space after
  {\addtolength{\@totalleftmargin}{2em}
   \addtolength{\linewidth}{-2em}
   \parshape 1 2em \linewidth}% body font
  {}% indent
  {\bfseries}% header font
  {.}% punctuation
  {.5em}% after theorem header
  {}% header specification (empty for default)
\makeatother

%\newtheorem{theorem}{Theorem}
\theoremstyle{researchTopic}
\newtheorem{theorem}{Research Topic}

\theoremstyle{indented}
% \newtheorem{corollary}{Question}
\newtheorem{corollary}{Question}[theorem]

\newtheorem{lemma}[theorem]{Lemma}

\newcounter{question}
\newenvironment{question}[1][]{\refstepcounter{question}\par\medskip
  \textbf{Q\thequestion. #1} \rmfamily}{\medskip}

% here we define how chapter titles look like, see manual to 
% "The titlesec , titleps and titletoc Packages"
\titleformat{\chapter}[display]
	{\centering \normalsize}
	{\MakeUppercase{\chaptertitlename} \thechapter}
	{0pc}
	{\normalsize \uppercase}



\titleformat{\section}[hang]{\bfseries \uppercase}
{\thesection}{12pt}{}

\titleformat{\subsection}[hang]{\bfseries}
{\thesubsection}{12pt}{}

\titleformat{\subsubsection}[hang]{\bfseries}
{\thesubsubsection}{12pt}{}

% Chapter name in TOC
\renewcommand{\cftchappresnum}{Chapter }
\renewcommand{\cftchapaftersnum}{:}
\renewcommand{\cftchapnumwidth}{6em}
\makeatletter
\newcommand*\updatechaptername{%
	\addtocontents{toc}{\protect\renewcommand*\protect\cftchappresnum{\@chapapp\ }}
}
\makeatother